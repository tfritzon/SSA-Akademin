\documentclass[a4paper,9pt,nobib]{tufte-book}
\usepackage[swedish]{babel}
\usepackage[utf8]{inputenc}
\usepackage{xspace}
\usepackage{graphicx}

\graphicspath{{graphics/}}

\hypersetup{colorlinks}% uncomment this line if you prefer colored hyperlinks (e.g., for onscreen viewing)

%%\IfFileExists{bergamo.sty}{\usepackage[osf]{bergamo}}{}% Bembo
%%\IfFileExists{chantill.sty}{\usepackage{chantill}}{}% Gill Sans

\usepackage{ifxetex}
\ifxetex
  \newcommand{\textls}[2][5]{%
    \begingroup\addfontfeatures{LetterSpace=#1}#2\endgroup
  }
  \renewcommand{\allcapsspacing}[1]{\textls[15]{#1}}
  \renewcommand{\smallcapsspacing}[1]{\textls[10]{#1}}
  \renewcommand{\allcaps}[1]{\textls[15]{\MakeTextUppercase{#1}}}
  \renewcommand{\smallcaps}[1]{\smallcapsspacing{\scshape\MakeTextLowercase{#1}}}
  \renewcommand{\textsc}[1]{\smallcapsspacing{\textsmallcaps{#1}}}
  \usepackage{fontspec}
\fi

\usepackage{lipsum}
\usepackage{booktabs}

\setkeys{Gin}{width=\linewidth,totalheight=\textheight,keepaspectratio}

\usepackage{fancyvrb}
\fvset{fontsize=\normalsize}

\newcommand{\hangp}[1]{\makebox[0pt][r]{(}#1\makebox[0pt][l]{)}}

\newcommand{\hangstar}{\makebox[0pt][l]{*}}


\newcommand{\vdqi}{\textit{VDQI}\xspace}
\newcommand{\ei}{\textit{EI}\xspace}
\newcommand{\ve}{\textit{VE}\xspace}
\newcommand{\be}{\textit{BE}\xspace}
\newcommand{\VDQI}{\textit{The Visual Display of Quantitative Information}\xspace}
\newcommand{\EI}{\textit{Envisioning Information}\xspace}
\newcommand{\VE}{\textit{Visual Explanations}\xspace}
\newcommand{\BE}{\textit{Beautiful Evidence}\xspace}

\newcommand{\TL}{Tufte-\LaTeX\xspace}

\newcommand{\monthyear}{%
  \ifcase\month\or January\or February\or March\or April\or May\or June\or
  July\or August\or September\or October\or November\or
  December\fi\space\number\year
}

\newcommand{\openepigraph}[2]{%
  %\sffamily\fontsize{14}{16}\selectfont
  \begin{fullwidth}
  \sffamily\large
  \begin{doublespace}
  \noindent\allcaps{#1}\\% epigraph
  \noindent\allcaps{#2}% author
  \end{doublespace}
  \end{fullwidth}
}

\newcommand{\blankpage}{\newpage\hbox{}\thispagestyle{empty}\newpage}

\usepackage{units}

\newcommand{\measure}[3]{#1/#2$\times$\unit[#3]{pc}}

\newcommand{\hlred}[1]{\textcolor{Maroon}{#1}}% prints in red
\newcommand{\hangleft}[1]{\makebox[0pt][r]{#1}}
\newcommand{\hairsp}{\hspace{1pt}}% hair space
\newcommand{\hquad}{\hskip0.5em\relax}% half quad space
\newcommand{\TODO}{\textcolor{red}{\bf TODO!}\xspace}
\newcommand{\ie}{\textit{i.\hairsp{}e.}\xspace}
\newcommand{\eg}{\textit{e.\hairsp{}g.}\xspace}
\newcommand{\na}{\quad--}% used in tables for N/A cells
\providecommand{\XeLaTeX}{X\lower.5ex\hbox{\kern-0.15em\reflectbox{E}}\kern-0.1em\LaTeX}
\newcommand{\tXeLaTeX}{\XeLaTeX\index{XeLaTeX@\protect\XeLaTeX}}
% \index{\texttt{\textbackslash xyz}@\hangleft{\texttt{\textbackslash}}\texttt{xyz}}
\newcommand{\tuftebs}{\symbol{'134}}% a backslash in tt type in OT1/T1
\newcommand{\doccmdnoindex}[2][]{\texttt{\tuftebs#2}}% command name -- adds backslash automatically (and doesn't add cmd to the index)
\newcommand{\doccmddef}[2][]{%
  \hlred{\texttt{\tuftebs#2}}\label{cmd:#2}%
  \ifthenelse{\isempty{#1}}%
    {% add the command to the index
      \index{#2 command@\protect\hangleft{\texttt{\tuftebs}}\texttt{#2}}% command name
    }%
    {% add the command and package to the index
      \index{#2 command@\protect\hangleft{\texttt{\tuftebs}}\texttt{#2} (\texttt{#1} package)}% command name
      \index{#1 package@\texttt{#1} package}\index{packages!#1@\texttt{#1}}% package name
    }%
}% command name -- adds backslash automatically
\newcommand{\doccmd}[2][]{%
  \texttt{\tuftebs#2}%
  \ifthenelse{\isempty{#1}}%
    {% add the command to the index
      \index{#2 command@\protect\hangleft{\texttt{\tuftebs}}\texttt{#2}}% command name
    }%
    {% add the command and package to the index
      \index{#2 command@\protect\hangleft{\texttt{\tuftebs}}\texttt{#2} (\texttt{#1} package)}% command name
      \index{#1 package@\texttt{#1} package}\index{packages!#1@\texttt{#1}}% package name
    }%
}% command name -- adds backslash automatically
\newcommand{\docopt}[1]{\ensuremath{\langle}\textrm{\textit{#1}}\ensuremath{\rangle}}% optional command argument
\newcommand{\docarg}[1]{\textrm{\textit{#1}}}% (required) command argument
\newenvironment{docspec}{\begin{quotation}\ttfamily\parskip0pt\parindent0pt\ignorespaces}{\end{quotation}}% command specification environment
\newcommand{\docenv}[1]{\texttt{#1}\index{#1 environment@\texttt{#1} environment}\index{environments!#1@\texttt{#1}}}% environment name
\newcommand{\docenvdef}[1]{\hlred{\texttt{#1}}\label{env:#1}\index{#1 environment@\texttt{#1} environment}\index{environments!#1@\texttt{#1}}}% environment name
\newcommand{\docpkg}[1]{\texttt{#1}\index{#1 package@\texttt{#1} package}\index{packages!#1@\texttt{#1}}}% package name
\newcommand{\doccls}[1]{\texttt{#1}}% document class name
\newcommand{\docclsopt}[1]{\texttt{#1}\index{#1 class option@\texttt{#1} class option}\index{class options!#1@\texttt{#1}}}% document class option name
\newcommand{\docclsoptdef}[1]{\hlred{\texttt{#1}}\label{clsopt:#1}\index{#1 class option@\texttt{#1} class option}\index{class options!#1@\texttt{#1}}}% document class option name defined
\newcommand{\docmsg}[2]{\bigskip\begin{fullwidth}\noindent\ttfamily#1\end{fullwidth}\medskip\par\noindent#2}
\newcommand{\docfilehook}[2]{\texttt{#1}\index{file hooks!#2}\index{#1@\texttt{#1}}}
\newcommand{\doccounter}[1]{\texttt{#1}\index{#1 counter@\texttt{#1} counter}}

%\documentclass[b5paper,twoside,openright]{book}
%\documentclass[a4paper,oneside,openright]{book}
\include{koncept/common}
%tlfällig fix för kompilering%
%%\nonstopmode

%%\headheight = 12.6pt

% Prepare for abstract
%%\pagestyle{plain}
%%\newenvironment{abstract}%
%%{\cleardoubblanklepage\null \vfill\begin{center}\bfseries Abstract \end{center}}%
%{\cleardoublepage\null \vfill\begin{center}%
%\bfseries \abstractname \end{center}}%
     {\vfill\null}

\usepackage{refbook}

% Prepare for index
%%\usepackage{makeidx}
%%\makeindex

\renewcommand{\subsubsection}[1]{\paragraph{#1}}

\newcommand{\smallfig}[4][1]{
  \begin{marginfigure}
    \includegraphics[width=#1\textwidth]{#2}
    \caption{#3}
    \label{#4}
    \vspace{-20pt}
  \end{marginfigure}
}

\newcommand{\mediumfig}[4][0.7]{
  \begin{figure}
    \centering
    \includegraphics[width=#1\textwidth]{#2}
    \caption{#3}
    \label{#4}
  \end{figure}
}

\newcommand{\largefig}[4][1]{
  \begin{figure*}
    \begin{center}
      \includegraphics[width=#1\textwidth]{#2}
      \caption{#3}
      \label{#4}
    \end{center}
  \end{figure*}
}

\newcommand{\smalltikz}[3]{
  \begin{marginfigure}
    #1
    \caption{#2}
    \label{#3}
  \end{marginfigure}
}
  
\usepackage{fontspec}
\setmainfont{ETBembo}

\begin{document}
% Frontpage
%\include{koncept/frontpage}

% Colophone

%% Tryckortssidan
%%\include{koncept/tryckort}

%\lfoot[\revision]{}
%\rfoot[]{\revision}

\cleardoublepage
%\pagestyle{fancy}

%\title{KONCEPT FÖR RADIOAMATÖRCERTIFIKAT}
%\author{Lennart Wiberg}
%\maketitle

\tableofcontents

%\setlength{\parindent}{0pt}
%\setlength{\parskip}{1ex plus 0.5ex minus 0.2ex}

\mainmatter

%%\include{koncept/errata}
%%\include{koncept/preface}
%%\include{koncept/foreword}
%%\include{koncept/introduction}

\chapter{Ellära}
\label{ellära}
\chapter{Ellära}
\label{ellära}

\section{Elektriska grundbegrepp}

\harec{a}{1.1}

Elektrisk laddning, spänning och ström hänger samman med hur materian är
uppbyggd.
Den förmåga ett material har att leda laddningar, det vill säga ström, kallas
konduktivitet.

\subsection{Grundämnen}
\marginnote{Fördjupning}
\index{grundämnen}

Det finns många former av materia.
Ofta är en form av materia sammansatt av andra former med enklare uppbyggnad.

Sammansatt materia kan sönderdelas på kemisk väg, men däremot inte de enklaste
formerna.
All materia är uppbyggd av atomer.
De enklaste materieformerna, som kallas \emph{grundämnen}, innehåller endast
ett slags atomer.
Över 100~grundämnen är kända.

Vart och ett av grundämnena har sin speciella atomuppbyggnad och därmed en
materialstruktur, som skiljer sig från varje annat grundämne.

Tre fjärdedelar av alla grundämnen är metaller (elektriska ledare) medan de
flesta övriga är icke-metaller (isolatorer).
Det finns även en liten mellangrupp som kallas för halvledare.

\subsection{Atomernas uppbyggnad}
\marginnote{Fördjupning}

Länge ansågs atomerna vara de minsta beståndsdelarna i materian.
Men omkring förra sekelskiftet upptäcktes att atomerna i sin tur består av ännu mindre
beståndsdelar, så kallade elementarpartiklar såsom protoner, neutroner, elektroner
med flera.
Det gemensamma namnet för alla dessa partiklar är \emph{nukleoner}.

En atom består dels av en kärna som är sammansatt av protoner och neutroner,
dels av elektroner, som kretsar omkring kärnan.

\begin{itemize}
\item Protonerna är positivt (+) laddade
\item Neutronerna är neutrala, ej laddade
\item Elektronerna är negativt (-) laddade.
\end{itemize}

%% \begin{marginfigure}%%{L}{0.5\textwidth}
%%   \includegraphics{images/cropped_pdfs/bild_2_1-01.pdf}
%%   \caption{Atomernas uppbyggnad}
%%   \label{fig:BildII1-1}
%%   \vspace{-20pt}
%% \end{marginfigure}

\smallfig{images/cropped_pdfs/bild_2_1-01.pdf}{Atomernas uppbyggnad}{fig:BildII1-1}

Elektronerna brukar förenklat beskrivas som att de kretsar i banor omkring atomkärnorna, 
liksom planeterna kretsar i banor omkring sina solar, vilket illustreras i 
bild \ref{fig:BildII1-1}.

Banor med samma avstånd till atomkärnan är på samma energinivå och sägs bilda
ett elektronskal. Det kan finnas mer än ett elektronskal. Ju fler elektroner 
som finns i ett elektronskal, desto starkare är elektronerna i skalet bundna
till atomen.
Det yttersta skalet kan emellertid aldrig innehålla fler än 8 elektroner.

Elektronerna i det yttersta skalet kallas för \emph{valenselektroner}, vilka
används även av angränsande atomer vid den kemiska bindningen till
atomstrukturer, molekyler och ämnen.
För bindningen behövs ett visst antal valenselektroner.

De valenselektroner som ej behövs för bindningen kan röra sig fritt genom
materia/strukturen.
De kallas fria elektroner och kan utgöra det vi kallar elektrisk ström.
Valenselektronerna är alltså inte bara av betydelse för materialets kemiska
struktur utan också för dess elektriska egenskaper.

Atomernas massa och volym är ytterst liten.
Tag som exempel en kub av koppar med volymen \SI{1}{\centi\metre\cubed} och vikten
\SI{8,9}{\gram}. 
Den består av ca \num{8,5d22} kopparatomer, dvs.
\num{85 000 000 000 000 000 000 000} stycken.
Fenomenet metallbindning gör att antalet fria elektroner i kuben är ungefär lika
med antalet atomer i den.

Varje elementarpartikel har en massa och en atoms totala massa är summan av
elementarpartiklarnas massor.
Den enklaste atomen är väteatomen med en proton och en elektron.
Väteatomens totala massa har kunnat beräknas till \SI{1,66d-24}{gram}.

Nästan hela massan i atomen är samlad till kärnans protoner och neutroner.
Var och en av dem har en massa som är ungefär \num{2000} gånger större än massan i en
elektron.

\subsection{Elektrisk laddning och kraftverkan}
\marginnote{Fördjupning}
\index{elektrisk laddning}

Enligt sägnen upptäckte Thales från Milteus redan för 2500 år sedan, att en bit
bärnsten drog till sig små grässtrån, sedan stenen gnidits mot en bit ylle.
Det grekiska ordet för bärnsten är ELEKTRON och de krafter som uppstod kom att
kallas ''elektriska''.
Av den elektriska spänningen mellan kroppar med olika laddning, verkar krafter
mellan dem och deras omgivning.
Krafterna kallas för elektriska fält och är det som gör att elektriskt laddade
kroppar kan komma i rörelse.

Ett exempel får man varje gång man kammar sig med en kam av isolerande material.
Då kommer håret att dras mot kammen därför att håret och kammen har
fått olika slags elektriska laddningar.
Samtidigt har hårstråna sinsemellan samma slags laddning och stöter bort
varandra --- håret ''reser sig''.

\begin{itemize}
	\item Lika laddningar stöter bort varandra.
	\item Olika laddningar drar varandra till sig.
\end{itemize}

\subsection{Konduktivitet --- Ledare, halvledare och isolator}
\marginnote{HAREC a.\ref{HAREC.a.1.1.1}\label{myHAREC.a.1.1.1}}
\index{konduktivitet}

En elektrisk ström sägs flyta, när de fria laddningsbärarna i ett material --- en
strömledare --- fås att röra sig samtidigt i samma riktning.
Hur många som rör sig beror på strömledarens egenskaper och spänningen mellan
ledarens ändar.

Alla material har någon grad av elektrisk ledningsförmåga som beror på
materialets atomstruktur, dimensioner och temperatur.
Vissa material (t.ex. metaller, kol, halvledare) leder elektrisk ström bättre
än andra (t.ex. glas, gummi, plast).
Mängden av fria laddningsbärare i materialet begränsar hur stor strömmen kan
bli.

\subsubsection{Ledare}
\index{ledare}
\index{konduktivitet!ledare}

Metaller har god elektrisk ledningsförmåga och kallas ledare.
Bäst ledande är de metaller, vars atomer har det minsta antalet
valenselektroner i det yttersta elektronskalet.
Koppar-, silver- och guldatomerna har en enda valenselektron och därmed mycket
god ledningsförmåga.
Järn, zink och magnesium har två valenselektroner och därmed något sämre
ledningsförmåga.
Ännu sämre ledare är de så kallade halvledarna med 3 till 5 valenselektroner.

\subsubsection{Isolatorer}
\index{isolator}
\index{konduktivitet!isolator}

Glas, plast, porslin och vissa mineraler har mycket dålig ledningsförmåga och
kallas isolatorer.
Isolatorerna är dåliga ledare på grund av att de har många valenselektroner i
sitt yttersta skal.
Maximalt ryms 8 valenselektroner.

I icke ledande material är elektronerna mycket hårt bundna till sitt valensskal
och därför svåra att flytta.
I fasta material är också positiva laddningar svåra att flytta, eftersom de är
bundna i atomkärnorna.
Atomerna är i sin tur bundna i en struktur som kännetecknar vart och ett
material.

\subsubsection{Halvledare}
\index{halvledare}
\index{konduktivitet!halvledare}
\index{halvledare!intrinsisk}
\index{halvledare!dopning}
\index{halvledare!ledningsförmåga}

Några grundämnen har en elektrisk ledningsförmåga som ligger mellan gränsvärdena
för att kallas elektriska ledare eller isolatorer.
Dessa ämnen tillhör gruppen halvledare och har en elektrisk ledningsförmåga som
varierar med ämnets struktur, renhet och temperatur.

En ren kristall av mineralen germanium [Ge] eller av kisel [Si] bildar ett
kristallgitter där atomerna binds till varandra med kovalenta bindningar.
Ämnena delar sina fyra valenselektroner med fyra andra atomer så att det
bildas en full oktett med åtta elektroner i valensskalet.

Då valensskalet innehåller åtta elektroner är det fullt, det finns inga fria
elektroner och ämnet leder inte elektrisk ström.
Båda dessa mineraler kan därför i denna form ses som isolatorer.
(\emph{intrinsisk halvledare})

Om några atomer av ett främmande material som till exempel arsenik, antimon,
indium eller gallium blandas in, (\emph{dopas in}), i kristallstrukturen så
förändras egenskaperna och den elektriska ledningsförmågan ökar tusenfalt.

\subsubsection{N-ledning}
\index{halvledare!N-ledning}

Man talar om N-ledande material respektive N-ledning- ''elektronledning''.

Germanium, kisel m.fl. halvledare har fyra elektroner med ''fasta platser'' i
valensskalet --- förutsatt att materialet är helt rent.
Då finns det inga fria elektroner för laddningstransport.

För att skapa fria elektroner kan det rena materialet förorenas -- dopas -- med
atomer av till exempel arsenik [As] eller antimon [Sb].

Båda dessa material är 5-värdiga.
De har 5 elektroner i valensskalet 4 elektroner är fast bundna medan
den 5:e är löst bunden till atomen.
Den 5:e elektronen kan lossgöras från atomen med yttre kraft, till exempel värme eller
elektrisk spänning och då skapas en fri elektron.
När en spänning läggs på materialet kommer den fria elektronen att vandra mot
den positiva polen.
Materialet är N-ledande.

\subsubsection{P-ledning -- ''hålledning''}
\index{halvledare!P-ledning}

När germanium eller kisel dopas med indium [In] eller gallium [Ga] blir de
P-ledande.
Indium och gallium är 3-värdiga --- deras valensskal innehåller 3 elektroner.
Men för en fast bindning med germanium eller kisel saknas det en elektron och
det uppstår då ett ''hål'' --- en ''bristelektron''.
Hålet kan fyllas ut av en elektron från en annan atom.
I den atom som elektronen lämnar bildas det i sin tur ett hål osv.
När en spänning läggs på, kommer ''hålet'' att vandra mot den negativa polen.
Materialet är då P-ledande.

\subsection{Elektrisk spänning --- Enheten volt}
\marginnote{HAREC a.\ref{HAREC.a.1.1.2}\label{myHAREC.a.1.1.2b},
  a.\ref{HAREC.a.1.1.3}\label{myHAREC.a.1.1.3b}}
\index{elektrisk spänning}
\index{spänning}
\index{volt (V)}
\index{enheter!volt (V)}
\index{symbol!\(U\) spänning}
\index{symbol!\(V\) spänning}

%%Bild \ref{fig:BildII1-2} illustrerar ett tankeförsök med ett rör med kulor i. 

%% \begin{figure*}
%% \begin{center}
%% \includegraphics[width=\textwidth]{images/cropped_pdfs/bild_2_1-02.pdf}
%% \caption{Tankeförsök med kulor i ett rör}
%% \label{fig:BildII1-2}
%% \end{center}
%% \end{figure*}

%%\largefig{images/cropped_pdfs/bild_2_1-02.pdf}
%%         {Tankeförsök med kulor i ett rör}{fig:BildII1-2}

Bild \ref{fig:BildII1-2} illustrerar ett tankeförsök med ett rör med kulor i.
Materialet i röret tänks motsvara atomstrukturen i en strömledare och kulorna
de fria elektronerna.
Tänker man sig ett slag mot en ände av röret så flyttar det sig av den energi
som tillförs.
På grund av obundenheten till röret följer av masströgheten kulorna inte med
röret, utan hamnar i dess ena ände.

%% \begin{figure}
%% 	\centering
%% 	\includegraphics[width=0.7\textwidth]{images/cropped_pdfs/bild_2_1-02.pdf}
%% 	\caption{Tankeförsök med kulor i ett rör}
%% 	\label{fig:BildII1-2}
%% \end{figure}

\mediumfig[0.85]{images/cropped_pdfs/bild_2_1-02.pdf}
          {Tankeförsök med kulor i ett rör}{fig:BildII1-2}

Att kulorna samlas i ena änden av röret tänks motsvara ett elektronöverskott i
ena änden av en ledare och ett motsvarande underskott i den andra änden.

Man kallar änden med elektronöverskott för minuspol och änden med
elektronunderskott för pluspol.
Olika stora elektriska laddningar vid polerna innebär att de sinsemellan har
olika potential.
Potentialskillnaden kallas spänning.

{\em Likspänning} innebär ett överskott av elektroner och alltid vid samma
anslutningspol.\\
{\em Växelspänning} innebär ett överskott av elektroner, omväxlande vid den ena
anslutningspolen och den andra.\\
Måttenheten för spänning är \unit{volt} som betecknas \unit{[V]}.
I formler betecknas spänning med

\begin{itemize}
  \item \(U\) för effektivvärdet
  \item \(u\) för momentanvärdet (ögonblicks-)
  \item \(\hat{u}\) för toppvärdet (amplitud-).
\end{itemize}

Bild \ref{fig:BildII1-16} i avsnitt 1.6 illustrerar relationen mellan värdena
för en sinuskurva.

Bild \ref{fig:BildII1-16} illustrerar relationen mellan dessa för en 
sinuskurva.

Spänningen över ändpunkterna på en strömledare är 1 volt [V], då
ledaren genomflyts av en likström av 1 ampere [A] under
effektutvecklingen 1 watt [W].

\subsection{Symboler}

%% %\begin{wrapfigure}[6]{R}{0.4\textwidth}
%% \begin{marginfigure}
%% %  \begin{mdframed}
%%     \begin{circuitikz}
%%       \draw
%%       (3,1) to[battery1] (1,1)
%%       ;
%%     \end{circuitikz}
%%     \caption{Schemasymbol för batteri}
%%     \label{fig:bildII2-batteri}
%% %  \end{mdframed}
%% \end{marginfigure}
%% %\end{wrapfigure}

\smalltikz{
  \begin{circuitikz}
    \draw
    (3,1) to[battery1] (1,1)
    ;
  \end{circuitikz}
}{Schemasymbol för batteri}{fig:bildII2-batteri}

\marginnote{Fördjupning}

När man ritar scheman för elektriska kretsar används symboler.
Symbolen i bild \ref{fig:bildII2-batteri} visar ett elektriskt batteri med en
enda cell.

Förtydligande kommentarer och skrivtecknen invid symbolen förekommer.
Ofta refererar dessa till en komponentlista.
Se även kapitel \ref{komponenter}.

\subsection{Elektrisk ström --- Enheten ampere}
\marginnote{HAREC a.\ref{HAREC.a.1.1.2}\label{myHAREC.a.1.1.2a}, a.
	\ref{HAREC.a.1.1.3}\label{myHAREC.a.1.1.3a}}
\index{elektrisk ström}
\index{ström}
\index{ampere (A)}
\index{enheter!ampere (A)}
\index{symbol!\(I\) ström}
När en sluten strömkrets innehåller en spänningskälla, kan en
laddningsutjämning ske genom kretsen.
Det innebär att fria elektroner förflyttar sig genom kretsen i riktning från
spänningskällans minuspol till dess pluspol.
Vid pluspolen är det nämligen brist på negativa laddningar och naturen söker
alltid en utjämning.
Under utjämningsförloppet är spänningskällan även en strömkälla.

I gaser och elektrolyter (elektriskt ledande vätskor och geler) samt i
halvledare består strömmen av joner (positiva eller negativa
laddningar), i metaller däremot av elektroner (negativa laddningar).

Av tradition anses strömriktningen vara positiv i jonströmmens
riktning -- den så kallade tekniska strömriktningen -- medan
elektronströmmens riktning är den motsatta -- den så kallade
fysikaliska strömriktningen.

Måttenheten för ström är \emph{ampere} \(\unit{[A]}\) \cite{SIbrochure8}.

I formler betecknas ström med
\(I\) för effektivvärdet,\\
\(i\) för momentanvärdet (ögonblicks-),\\
\(\hat{i}\) för toppvärdet (amplitud-).

Strömmen är \SI{1}{A} när \num{6,25d18} elektroner per sekund
flyter genom ett givet ledartvärsnitt, vilket motsvarar laddningen
\SI{1}{coloumb}.

\subsection{Strömkrets}
\smallfig{images/cropped_pdfs/bild_2_1-03.pdf}
         {Potential och spänning i en strömkrets}
         {fig:BildII1-3}
\index{strömkrets}
Bild \ref{fig:BildII1-3} visar potential och spänning i en strömkrets.
En elektrisk strömkrets består av en eller flera energikällor och
energiförbrukare.
Källor kan vara batterier, nätaggregat etc.
Förbrukare kan vara lampor, ledningar etc.
Varje energiförbrukare har en resistans och de elektriska laddningarna ''köar''
före förbrukaren, strax efter förbrukaren finns ingen kö.
Det uppstår en skillnad i laddningsmängd (en potentialskillnad) mellan varje
punkt i en strömkrets, när det flyter ström.
Man talar om spänningsfall.

%% \begin{marginfigure}
%% \includegraphics{images/cropped_pdfs/bild_2_1-03.pdf}
%% \caption{Potential och spänning i en strömkrets}
%% \label{fig:BildII1-3}
%% \end{marginfigure}

\subsection{Strömförlopp}
\marginnote{Fördjupning}
\index{strömförlopp}

Likströms- och växelströmsförloppen kan vara sammansatta av ett huvudförlopp
och underordnade förlopp.

Likström kan ha konstant styrka eller den kan variera enligt något förlopp,
men växlar aldrig riktning.

Växelström kan variera enligt något visst förlopp, till exempel sinusvåg,
fyrkantsvåg, och växlar ständigt riktning.

\subsection{Resistans --- Enheten ohm}
\marginnote{HAREC a.\ref{HAREC.a.1.1.2}\label{myHAREC.a.1.1.2c}, 
	a.\ref{HAREC.a.1.1.3}\label{myHAREC.a.1.1.3c}}
\index{resistans}
\index{ohm (\(\Omega\))}
\index{enheter!ohm (\(\Omega\))}
\index{symbol!\(R\) resistans}

När fria elektroner tvingas fram genom atomstrukturen i en ledare, till exempel
glödtråden i en lampa, så avgår energi i form av värme.
Detta fenomen kallas för resistans (av latinets resistere som betyder att
motstå).
Resistansen och därmed förlusterna i en strömkrets fördelas i
förhållande till de ingående materialen och deras dimensionering.

Resistans uttrycks i enheten \emph{ohm} \cite{SIbrochure8} och betecknas med
den grekiska bokstaven omega (\(\Omega\)).

I formler betecknas resistansen i en elektrisk krets eller en del av den med
\(R\).

Resistansen i en resistor är \SI{1}{\Omega}, när en spänning av \SI{1}{V}
driver en ström av \SI{1}{A} genom den resistorn.

\subsection{Ohms lag}
\marginnote{HAREC a.\ref{HAREC.a.1.1.4}\label{myHAREC.a.1.1.4}}
\index{Ohms lag}
\index{resistor!Ohms lag}

Ohms lag beskriver sambandet mellan grundbegreppen ström
\(I\ \unit{[ampere]}\), spänning \(U\ \unit{[volt]}\) och resistans
\(R\ \unit{[ohm]}\).
Sambandet gäller både för likspänning och effektivvärdet för växelspänning och
växelström.

I en ledare med resistansen \(R\) är strömstyrkan \(I\) genom resistansen
proportionell mot den pålagda spänningen \(U\).

\begin{equation*}
  U = I \cdot R \qquad
  I = \frac{U}{R} \qquad
  R = \frac{U}{I}
\end{equation*}

\subsection{Kirchhoffs lagar}
\marginnote{HAREC a.\ref{HAREC.a.1.1.5}\label{myHAREC.a.1.1.5}}
\index{Kirchhoffs lagar}
\index{Kirchhoffs strömlag}
\index{Kirchhoffs spänningslag}

Den tyske fysikern G. R. Kirchhoff (1824--1887) formulerade sina välkända lagar
först 1845 och sedan 1847.

Kirchhoffs strömlag:

Den algebraiska summan av alla strömmar (tagna med tecken) som flyter
till eller från varje punkt i en elektrisk krets, är lika med noll.

\begin{equation*}
  I_1 + I_2 + I_3 + \cdots + I_n = 0
\end{equation*}

Kirchhoffs spänningslag:

I varje sluten strömkrets är den algebraiska summan av alla spänningskällor lika
med det totala spänningsfallet i alla resistorer.

Uttryckt på ett annat sätt är algebraiska summan av spänningarna i en
strömkrets (tagna med tecken) lika med noll.

\begin{equation*}
  U_1 + U_2 + U_3 + \cdots + U_n = 0
\end{equation*}

\subsection{Elektrisk effekt --- Enheten watt}
\marginnote{HAREC a.\ref{HAREC.a.1.1.6}\label{myHAREC.a.1.1.6}, 
	a.\ref{HAREC.a.1.1.7}\label{myHAREC.a.1.1.7}}
\index{elektrisk effekt}
\index{effekt}
\index{voltampere (VA)}
\index{enheter!voltampere (VA)}
\index{watt (W)}
\index{enheter!watt (W)}
\index{symbol!\(P\) effekt}

När en ström flyter genom en resistans utvecklas värme.
Värme är en form av effekt, som är högre ju starkare strömmen och högre
spänningen är.

Måttenheten \emph{voltampere} [VA] för elektrisk effekt härleds ur
produkten av volt [V] och ampere [A].

För effekt som alstras av likström används enheten \emph{watt} [W]
\cite{SIbrochure8} i stället för \emph{voltampere} [VA].
Vid sidan om grundenheten \SI{1}{W} används delar och multipler av
denna.

\begin{equation*}
	\SI{1}{volt} \unit{[U]} \cdot \SI{1}{ampere [I]} = \SI{1}{watt [P]}
\end{equation*}

Effektformeln \(P = U \cdot I\) gäller i första hand för likström men även för
växelström om belastningen är resistiv och ström och spänning inte är
fasförskjutna.
Formeln kan för att underlätta beräkningar skrivas om på flera sätt.

Vi börjar med att lösa ut \(I\) ur Ohms lag \(U = R \cdot I\) 

\[
	I = \frac{U}{R}
\]

Vi sätter sedan in uttrycket för \(I\) i effektformeln

\[
	P=U \cdot I \qquad \Rightarrow \qquad P= \dfrac{U \cdot U}{R} \qquad
	\Rightarrow \qquad P= \dfrac{U^2}{R}
\]

På motsvarande sätt kan vi ersätta \(U \quad\text{med}\quad R \cdot I\)

\[ \begin{array}{lllll}
P=U \cdot I & \Rightarrow & P = R \cdot I \cdot I  & \Rightarrow & P = R \cdot I^2
\end{array} \]


Med hjälp av dessa formler kan effekten beräknas ur resistans- och strömvärdena
respektive ur resistans- och spänningsvärdena.
För övriga formler se formelsnurran bild \ref{fig:BildII1-4}

\subsection{Elektrisk arbete --- Enheten joule}
\textbf{FÖRDJUPNING}
\index{elektriskt arbete}
\index{joule (J)}
\index{enheter!joule (J)}
\index{symbol!\(W\) energi, arbete}

Energi finns i olika former, alltid och överallt.
Energi kan varken skapas eller förstöras, bara omvandlas från en form till en
annan.
Formen kan vara mekanisk, kemisk, elektrisk etc.

Arbete är omvandlingsprocessen från en energiform till en annan.

Arbetsmängden i alla energiformer kan mätas med samma enhet \emph{joule}
\(\mathrm{[J]}\) \cite{SIbrochure8} och anges med symbolen \(W\) för Work.

\(1\ \mathrm{joule}\) motsvarar det arbete som utvecklas när ett föremål
förflyttas \(1\ \mathrm{meter}\) med kraften \(1\ \mathrm{newton\ [N]}\),
d. v. s. \(1\ \mathrm{newtonmeter\ [Nm]}\).

\(W = l \cdot F \ \ \ \mathrm{[J] = [Nm]}\)

Arbetet \(W\ \mathrm{[J]}\) är mer ju längre tid \(t\ [s]\) en viss effekt
\(P\ \mathrm{[W]}\) utvecklas.

\(W = t \cdot P \ \ \ \mathrm{[J] = [sW]}\)
  
\subsection{Joules lag}
\marginnote{HAREC a.\ref{HAREC.a.1.1.8}\label{myHAREC.a.1.1.8}}
\index{Joules lag}

\(Arbete\ =\ Effekt\ \cdot\ tid\ \ \ \mathrm{[W]} = \mathrm{[P]} \cdot \mathrm{[s]}\)

Eftersom effekten uttrycks som \(P = U \cdot I\) kan det elektriska arbetet
uttryckas som \(W = U \cdot I \cdot t\), vilket också är Joules lag.

Om grundenheterna för volt [U], ampere [I] och
sekund [s] sätts in i formeln fås en måttenhet, uttryckt som
voltamperesekunder [VAs] eller wattsekunder [Ws] eller joule\ [J].

Måttenheten för elektriskt arbete är 1 joule per sekund, som vanligen
kallas 1 wattsekund [1 Ws].
Vid sidan av grundenheten används multipler av denna.

Exempel:

\begin{align*}
  \centering
  \num{1}\,&\mathrm{kilowattsekund} &=& \SI{1}{kWs} &=& \SI{1000}{Ws} &=& \SI{1,0d3}{Ws} \\
  \num{1}\,&\mathrm{wattimme}       &=& \SI{1}{Wh}  &=& \SI{3600}{Ws} &=& \SI{3,6d3}{Ws} \\
  \num{1}\,&\mathrm{kilowattimme}   &=& \SI{1}{kWh} &=& \SI{1000}{Wh} &=& \SI{3,6d6}{Ws} \\
\end{align*}

\subsection{Formelsnurran}
\index{formelsnurran}
\marginnote{Fördjupning}

Så här finner man rätt formel i ''snurran'' (bild \ref{fig:BildII1-4}):
Välj ett segment med önskad storhet \(I\), \(U\), \(R\) eller \(P\) som det
första ledet i formeln.
Inom valt segment finns tre alternativ för det andra ledet i formeln.
Välj det alternativ som innehåller två kända storheter.

%% %\begin{figure*}[ht]
%% \begin{marginfigure}
%%   %%\begin{wrapfigure}{R}{0.3\textwidth}
%%   \includegraphics[width=\textwidth]{images/cropped_pdfs/bild_2_1-04.pdf}
%%   \caption{''Formelsnurra'' för Ohms och Joules lagar}
%%   \label{fig:BildII1-4}
%%   %  \vspace{-100pt}
%%   %%\end{wrapfigure}
%% \end{marginfigure}
%% %\end{figure*}

\smallfig{images/cropped_pdfs/bild_2_1-04.pdf}
         {''Formelsnurra'' för Ohms och Joules lagar}
         {fig:BildII1-4}
         
Bild \ref{fig:BildII1-4} visar ''Formelsnurra'' för Ohms och Joules lagar.

Exempel:

\subsubsection{Ohms lag}

\(R\) söks, \(U\) och \(I\) är kända;
Om \(U = \SI{230}{V}\) och \(I = \SI{2}{A}\), så blir

\begin{equation*}
  R=\dfrac{U}{I}=\dfrac{230}{2}=\SI{115}{\ohm}
\end{equation*}

\subsubsection{Joules lag}

\(P\) söks, \(U\) och \(I\) är kända;

Om \(U = \SI{230}{V} \) och \(I = \SI{2}{A} \), så blir

\begin{equation*}
  P = U \cdot I = 230 \cdot 2 = \SI{460}{\watt}
\end{equation*}

\subsection{Amperetimmar (Ah) och batterikapacitet}
\marginnote{HAREC a.\ref{HAREC.a.1.1.9}\label{myHAREC.a.1.1.9}}
\index{amperetimmar (Ah)}
\index{batterikapacitet}
\index{batteri}
\index{batteri!primärärbatteri}
\index{batteri!primärärcell}
\index{ackumulator}
\index{ackumulator!sekundärcell}
\index{ackumulator!sekundärbatteri}
\index{polspänning}
\index{elektrisk cell}

Det finns flera sätt att lagra energi.
Ett sätt är att göra det i kemisk form i speciella celler, där man kan ta ut
energin i elektrisk form.

Det finns celler som kan laddas upp och laddas ur upprepade gånger.
Sådana celler kallas vanligtvis ackumulator, sekundärbatteri eller
sekundärcell.

Det finns också sådana celler som endast kan användas en gång och som normalt
inte kan laddas upp igen.
Sådana celler kallas vanligtvis primärcell eller primärbatteri.

Energi i form av en elektrisk laddning kan även lagras i en kondensator.
Energin kan då lagras och tas ut utan omvandling.

Kapaciteten i en elektrisk cell uttrycks som produkten av den ström
\(\unit{[A]}\) som cellen avger och under den tid \(\unit{[s, h]}\) detta
kan ske.
Uttryckt med tidsenheten timmar blir då kapaciteten \(\unit{[Ah]}\).

Den kapacitet som anges i en cells produktdata är den nominella.
Denna kapacitet gäller endast under vissa normerade förhållanden såsom
celltemperatur, strömstyrka och urladdningstid.

Den praktiska kapaciteten i en cell begränsas av användningen.
En elektrisk cell avger sålunda regelmässigt mindre energimängd, ju högre
urladdningsströmmen är.
Kapaciteten i en elektrisk cell skiljer sig i det avseendet från den i
till exempel en oljetank, där man kan ta ut lika mycket energimängd som man häller i
och oberoende av hur fort man gör det.

Elektriska celler kan samlas till så kallade batterier, varvid cellerna oftast
seriekopplas.
Batteriets polspänning är då summan av cellernas polspänningar.

Hur stort arbete ett batteri avger, beror såväl på hela batteriets
polspänning som på de enskilda cellernas kapacitet.

Exempel:

Ett batteri med polspänningen \SI{12}{V} och cellkapaciteten
\SI{100}{Ah} kan nominellt avge
\(P = U \cdot I = 12 \cdot 100 = \SI{1200}{VAh} = \SI{1,2}{kWh}\).


Hur länge batteriet ''räcker'' per laddning beror på vilken
strömstyrka man tar ut.
Tar man ut \SI{1}{A} ur \SI{100}{Ah}-cellen här ovan, så blir
urladdningstiden nominellt \(t = \SI{100}{Ah} / \SI{1}{A} = \SI{100}{h}\).
  % Kollad: Anders II
% \include{koncept/chapter1-2}  % Kollad: Anders II
% \include{koncept/chapter1-3}  % Kollad: Anders II
% \include{koncept/chapter1-4}  % Kollad: Anders II
% \include{koncept/chapter1-5}  % Kollad: Anders II
% \include{koncept/chapter1-6}  % Kollad: Anders II
% \include{koncept/chapter1-7}  % Kollad: Anders II
% \include{koncept/chapter1-8}  % Kollad: Anders II
% \include{koncept/chapter1-9}  % Kollad: Anders II
% \include{koncept/chapter1-10} % Kollad: Anders II

% \chapter{Komponenter}
% \label{komponenter}
% \include{koncept/chapter2-1}  % Kollad: Anders II
% \include{koncept/chapter2-2}  % Kollad: Anders II
% \include{koncept/chapter2-3}  % Kollad: Anders II
% \include{koncept/chapter2-4}  % Kollad: Anders II
% \include{koncept/chapter2-5}  % Kollad: Anders II
% \include{koncept/chapter2-6}  % Kollad: Anders II
% \include{koncept/chapter2-7}  % Kollad: Anders II
% \include{koncept/chapter2-8}  % Kollad: Anders II
% \include{koncept/chapter2-9}  % Kollad: Anders II
% \include{koncept/chapter2-10} % Kollad: Anders II

% \chapter{Kretsar}
% \include{koncept/chapter3-1} % Kollad: Anders I
% \include{koncept/chapter3-2} % Kollad: Anders I
% \include{koncept/chapter3-3} % Kollad: Anders I
% \include{koncept/chapter3-4} % Kollad: Anders I
% \include{koncept/chapter3-5} % Kollad: Anders I
% \include{koncept/chapter3-6} % Under layouting... Anders
% \include{koncept/chapter3-7}
% \include{koncept/chapter3-8}
% \include{koncept/chapter3-9}
% \include{koncept/chapter3-10}

%\chapter[Isolation och jord]{Isolation och jordning}
% \include{koncept/chapter4-1}
%
%\chapter{Mottagare}
% \include{koncept/chapter5-1}
% \include{koncept/chapter5-2}
% \include{koncept/chapter5-3}
% \include{koncept/chapter5-4}
% \include{koncept/chapter5-5}
% \include{koncept/chapter5-6}
% \include{koncept/chapter5-7}
% \include{koncept/chapter5-8}
% \include{koncept/chapter5-9}
%
%\chapter[Sändare]{Sändare och transceivers}
% \include{koncept/chapter6-1}
% \include{koncept/chapter6-2}
%
%\chapter{Antennsystem}
% \include{koncept/chapter7-1}
% \include{koncept/chapter7-2}
% \include{koncept/chapter7-3}
% \include{koncept/chapter7-4}
% \include{koncept/chapter7-5}
% \include{koncept/chapter7-6}
%
%\chapter{Vågutbredning}
% \include{koncept/chapter8-1}
% \include{koncept/chapter8-2}
% \include{koncept/chapter8-3}
% \include{koncept/chapter8-4}
% \include{koncept/chapter8-5}
% \include{koncept/chapter8-6}
% \include{koncept/chapter8-7}
%
%\chapter{Mätteknik}
% \include{koncept/chapter9-1}
% \include{koncept/chapter9-2}
%
%\chapter{EMC}
% \include{koncept/chapter10-1}
% \include{koncept/chapter10-2}
% \include{koncept/chapter10-3}
% \include{koncept/chapter10-4}
%
%\chapter{Elektromagnetiska fält}
% \include{koncept/chapter11-1}
%
%\chapter{Elsäkerhet}
% \include{koncept/chapter12-1}
% \include{koncept/chapter12-2}
% \include{koncept/chapter12-3}
% \include{koncept/chapter12-4}
%
%\include{koncept/part3}
%
%\chapter{Trafikreglemente}
% \include{koncept/chapter13-1}
% \include{koncept/chapter13-2}
% \include{koncept/chapter13-3}
% \include{koncept/chapter13-4}
% \include{koncept/chapter13-5}
% \include{koncept/chapter13-6}
%
%\chapter{Bestämmelser}
% \include{koncept/chapter14-1}
% \include{koncept/chapter14-2}
% \include{koncept/chapter14-3}
%
%\chapter{Regler och trafikmetoder}
% \include{koncept/chapter15-1}
%
%\chapter{Morsesignalering}
% \include{koncept/chapter16-1}
%
% \appendix
%
% \include{koncept/appendix-a}
% \include{koncept/appendix-b}
% %\include{koncept/appendix-c}
% \include{koncept/appendix-d}
% \include{koncept/appendix-e}
% \include{koncept/appendix-f}
% \include{koncept/appendix-g}
% \include{koncept/appendix-h}
% \include{koncept/appendix-i}
% \include{koncept/appendix-j}
% \include{koncept/appendix-l}
% \include{koncept/appendix-m}


%%\listoffigures
%%\listoftables

%%\onecolumn
%%\bibliography{koncept.bib}
%%\bibliographystyle{plain}

\backmatter

%%\printindex
%%\include{koncept/backpage}

\end{document}
