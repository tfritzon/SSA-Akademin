\section{Bestämmelser för nödtrafik}



\paragraph{Nödtrafik} \hfill \\

Trafik för att förebygga eller reducera ytterligare
skadeverkningar vid uppenbar risk för allvarlig person- eller
egendomsskada, eller då någon är svårt skadad eller i livsfara.

\paragraph{Nöd bryter lag} \hfill \\

Ingen bestämmelse i reglementet får utgöra hinder för en landstation
att under utomordentliga förhållanden begagna sig av alla till buds
stående medel för att bringa hjälp åt en nödställd rörlig station.


\paragraph{Nödsignaler vid sjö- och luftradiotrafik} \hfill \\

På telefoni utgörs nödsignalen av ordet MAYDAY (uttalas som det
franska ordet m'aider [mädé] och inte som [mäjdäj]).

Denna nödsignal tillkännager att ett fartyg, luftfartyg eller
färdmedel av annat slag hotas av allvarlig och överhängande fara och
begär omedelbar hjälp.

\paragraph{Internationella nödfrekvenser} \hfill \\

Utsändning av nödsignaler på telefoni på VHF-banden sker i första hand
på frekvenserna \SI{121,5}{MHz} för flyg och på kanal 16 VHF Marin på
\SI{156,8}{MHz}. Tidigare har kanal 16 på privatradiobandet också
använts för nödsignalering men har i dag ingen officiell status.

Utsändning av nödsignaler på HF (kortvåg) kan ske på vilken frekvens
som helst. 


\section{Åtgärder då nödsignaler uppfattas}

När man uppfattar ett nödanrop så är det viktigt att först vänta och
höra om någon annan station svarar.

\begin{tabularx}{\columnwidth}{lX}
  \textbf{Avbryt} & Avbryt egen lyssning \\\hline
  \textbf{Lyssna} & Avlyssna och skriv ner nödmeddelandet, position,
  frekvens med mera som är av vikt. \\\hline
  \textbf{Larma} & Kvitteras inte nödsignalen från mark- eller
  kustradiostation, ring 112, invänta telefonistsvar och begär
  Flygräddning eller Sjöräddning och meddela dina iakttagelser\\\hline
\end{tabularx}

Tänk på följande om du får kontakt med en nödställd:

\begin{tabularx}{\columnwidth}{lX}
  \textbf{Var?} & Var finns olycksplatsen? Position? \\ \hline
  \textbf{Vad?} & Vad har inträffat? \\ \hline
  \textbf{Vem?} & Vem är det som anropar? (Anropssignal, namn på
  fartyg) \\ \hline
  \textbf{Hur?} & Vad behöver den nödställde för hjälp? (Ambulans,
  medicin, brandkår, sjöräddning, flygräddning)\\ \hline
  \textbf{När?} & Notera tidpunkt och frekvens när du uppfattade
  nödmeddelandet\\ 
\end{tabularx}

\section{Nödsignalering på amatörradiobanden}

På amatörradiofrekvenser kan signalen \textbf{CQ EMERGENCY}
(nödmeddelande) förekomma som nödsignal vid nödsituationer på land när
allvarlig och överhängande fara hotar människoliv.

Nödsignalen bör åtföljas av ett nödmeddelande där det anges nödlägets
art, plats samt vilken hjälp som behövs.

Detta är dock INTE en officiellt erkänd nödsignal.

\textit{Vid nödanrop, följ åtgärdslistan i tillämpliga delar enligt
  föregående sida.}

\vspace{1em} \hrule \vspace{1em}

\emph{Kom ihåg:}

\begin{itemize}
\item MAYDAY
\item CQ EMERGENCY
\item När du uppfattar ett nödanrop så ädet viktigt att först vänta och
  lyssna om någon annan station svarar
  \itemÅtgärder: Avbryt, Lyssna, Larma 112
\end{itemize}

\emph{Lär dig:}

\begin{itemize}
\item  Vad du skall göra om du uppfattar ett nödanrop
\end{itemize}
