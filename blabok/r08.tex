CEPT är en samarbetsorganisation av europeiska Post- och Telekommunikationsmyndigheter.

Dess uppgift är att samordna och reglera elektronisk kommunikation inom Europa.
Detta sker genom att ta fram beslut, rekommendationer och rapporter.

För amatörradion finns det två rekommendationer, som kortfattat beskrivs i detta
kapitel.

\section{Rekommendation T/R 61-01}

\textbf{(Turistlicensen)}

CEPT-rekommendationen T/R 61-01
gör det möjligt för sändaramatörer från
CEPT:s medlemsländer att använda amatör-
radiosändare vid temporära besök i andra
CEPT-länder, utan att behöva ansöka om
tillfälligt amatörradiotillstånd.

\emph{\textbf{Anm.} Många länder tillämpar, att turist-
	licensen gäller i högst 3 månader.}

Sedan 1994 finns det också möjlighet för
länder utanför CEPT-sfären att ansluta sig
till denna rekommendation.

\emph\textbf{{Anm.} Vid denna boks tryckning är åtta
	länder anslutna, däribland USA.}

--- TODO: Kontrollera hur det är i dag ---

Det finns endast en licensklass i rekom-
mendationen. Det fordras inte kunskap
i telegrafi, för att få sända på kortvågs-
banden. Några länder kräver dock detta
som ett tillägg.

Varje medlemsland specificerar vilka
nationella licensklasser, som överensstämmer
med CEPT-klassen.

\emph{Alla svenska sändaramatörer är godkända
för CEPT-klassen.}

\section{Utdrag ur tillämpningsföreskrifter}

\begin{tabularx}{\columnwidth}{lX}
3.1 & Sändaramatören ska kunna visa upp
sitt amatörradiotillstånd, eller sitt
amatörradiocertifikat med påtecknad
anropssignal för myndigheten i det
land han/hon besöker.\vspace{1ex}\\
 & \emph{\textbf{Anm.} En notering måste finnas, att
 	rekommendationen T/R 61-01 uppfylls. }\vspace{1ex}\\
 
 3.2 & Det är inte nödvändigt att ta med sig
 sin egen radiostation. Det går också
 bra att låna en station i besökslandet.\vspace{1ex}\\
\end{tabularx}

%För layout
\begin{tabularx}{\columnwidth}{lX}
3.3 & När man sänder i besökslandet ska
CEPT-licensinnehavaren använda sin
nationella anropssignal, föregången av
besökslandets prefix.
Besökslandets prefix och den natio-
nella anropssignalen ska separeras av
tecknet ''/'' (vid telegrafi eller digitala
trafiksätt), eller ordet ”\textbf{streck}”,
engelska ”\textbf{stroke}” eller ”\textbf{slash}” (vid
telefoni).\vspace{1ex}\\

& \emph{\textbf{Anm.} Anropssignalens utformning
i respektive CEPT-land, se tabell på
nästa sida.}\vspace{1ex}\\

3.4 & Att använda CEPT-licensen i ett annat
land, innebär inte några rättigheter att
importera eller exportera radioutrustning, 
utöver de tullbestämmelser som
gäller i respektive land.\vspace{1ex}\\
\end{tabularx}

\emph{Vilka länder, som tillämpar CEPT:s
	rekommendationer T/R 61-01 och deras
	nationella krav och regler, finns i sin
	helhet att läsa på Internet och på SSA:s
	webbplats www.ssa.se under rubriken
	”Att köra radio utomlands”.}

\section{Rekommendation T/R 62-02 harmoniserad certifikatsnivå}

CEPT-licensen enligt Rekommendation
T/R 61-01 kan bara användas vid tillfälligt
besök i annat land.

Om det däremot är en permanent bosättning
kan rekommendation T/R 61-02 användas.
CEPT-länderna har kommit överens om att
samordna kompetenskraven, så att det ska
vara möjligt att erhålla en licens i ett annat
land, utan att avlägga nya prov.

Rekommendationen kallas för HAREC,
Harmonized Amateur Radio Examination
Certificate.

En notering på amatörradiocertifikatet,
att kompetenskraven uppfyller rekommendationen 
T/R 61-02, måste dock finnas.

I kapitel R5 - Anropssignaler 
beskrivs exempel på olika länders
identitetsprefix.

I detta kapitel beskrivs exempel på olika
länders prefix, som sätts före den egna
anropssignalen, vid gästbesök i respektive 
land. Detta prefix kan se annorlunda
ut, jämfört med landets identitetsprefix.

Vilka länder, som tillämpar CEPT:s
rekommendationer T/R 61-02, finns i sin
helhet att läsa på Internet och på SSA:s
webbplats www.ssa.se under rubriken
”Att köra radio utomlands”.

\section{Utdrag ur tabell T/R 61-01}

Prefix som skall användas \textit{före} egen signal vid besök utomlands

\begin{tabular}{ll}
	\textbf{CEPT-land} & \textbf{Prefix} \\
	Belgien            &  ON/\\
	Cypern             &  5B/\\
	Danmark            &  OZ/\\
	England            &  M/\\
	Finland            &  OH/\\
	Frankrike          &  F/\\
	Nederländerna      &  PA/\\
	Irland             &  EI/\\
	Italien            &  I/\\
	Norge              &  LA/\\
	Portugal           &  CT/\\
	Schweiz            &  HB9/\\
	Spanien            &  EA/, EB/\\
	Sverige            &  SM, SA/\\
	Tjeckien           &  OK/\\
	Turkiet            &  TA/\\
	Tyskland           &  DL/\\
	Åland              &  OHØ/\\
	Österrike          &  OE/
\end{tabular}

%\vspace{1em} \hrule \vspace{1em}
% Bara för layout
\newpage

\emph{Kom ihåg:}

\begin{itemize}
	\item Sändaramatörer från CEPT:s medlemsländer har möjlighet att använda
	amatörradiosändare vid temporära besök i annat CEPT-land
	
	\item Sändaramatör ska kunna visa upp sitt amatörradiotillstånd eller
	sitt amatörradiocertifikat i besöks	landet
	
	\item CEPT-licensen ger ingen rätt att importera eller exportera radiout-
	rustning
\end{itemize}


\emph{Lär dig:}

\begin{itemize}
	\item hur olika lönders CEPT-prefix ser ut
	\item att besökslandets prefix skall föregå din anropssignal
\end{itemize}

