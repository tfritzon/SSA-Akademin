\section{ITU -- Internationella teleunionen}

Vid \emph{internationell} amatörradiotrafik gäller följande reglemente:

\subsection{Utrag ur ITU RR artikel 25}

\begin{tabularx}{\columnwidth}{lX}
	25.1 & Radiokommunikation mellan amatör-
	stationer i olika länder ska vara
	tillåtet, såvida inte administrationen
	i ett av de berörda länderna har
	anmält att de motsätter sig sådan
	radiokommunikation.\vspace{1ex}\\
	
	25.2 & Sändningar mellan amatörstationer
	i olika länder ska vara begränsad till
	kommunikation, som överensstäm-
	mer med ändamålet med amatör-
	radio, enligt definitionen i No. 1.56\footnote{%
		Amatörradiotrafik:
		Icke yrkesmässig radiotrafik för
		övning, kommunikation och tekniska 
		undersökningar, bedriven
		i personligt intresse och utan
		vinningssyfte.}
	och till anmärkningar av personlig
	karaktär.\vspace{1ex}\\
	
	25.2A & Sändningar mellan amatörstationer
	i olika länder ska inte vara kodade
	för att dölja innehållet, utom för
	styrsignaler som utväxlas mellan
	jordkommandostationer och rymd-
	stationer inom amatörradiotjänsten\vspace{1ex}\\
	
	25.3 &Amatörstationer får användas för att
	sända internationell kommunikation
	för tredje part endast vid nödsitua-
	tioner och för katastrofhjälp. En
	administration får bestämma i vilken
	utsträckning detta får tillämpas för
	amatörstationer som omfattas av
	dess regelverk.\vspace{1ex}\\
	
	25.5 &Administrationer ska bestämma
	huruvida en person, som ansöker
	om licens för amatörradio, behöver
	visa färdighet i att sända och ta
	emot texter med morsesignaler.\vspace{1ex}\\
	
	25.6 & Administrationer ska bekräfta de
	operativa och tekniska kvalifikatio-
	nerna hos en person, som ansöker
	om att få använda en amatörradio-
	station.\vspace{1ex}\\



\end{tabularx}	

% Uppdelning av layoutskäl
\begin{tabularx}{\columnwidth}{lX}	
	
	25.7 & Den maximala effekten för en
	amatörradiostation ska anges av de
	aktuella administrationerna.\vspace{1ex}\\
		
	25.9 & Under sändningspassen ska
	amatörradiostationer sända sin
	anropssignal med korta intervall.\vspace{1ex}\\
		
	25.9A & Administrationer uppmuntras att
	vidta nödvändiga steg för att tillåta
	amatörradiostationer att förbereda
	sig för och uppfylla kommunika-
	tionsbehov som stöd för katastrof-
	insatser.\vspace{1ex}\\
	
	25.9B & En administration får bestämma
	huruvida man tillåter en person, som
	erhållit licens för amatörradio av en
	annan administration, att använda
	en amatörradiostation medan per-
	sonen tillfälligt befinner sig på dess
	territorium, med beaktande av de
	villkor eller begränsningar som
	anges.\vspace{1ex}\\
	
	25.10 & Villkoren enligt Sektion I i denna
	artikel ska även gälla, i tillämpliga
	delar, för amatörsatellittjänsten.\vspace{1ex}\\
	
	25.11 & Administrationer som tillåter rymd-
	stationer i amatörsatellittjänsten ska
	försäkra att tillräckligt med jord-
	kommandostationer etableras före
	uppskjutning, för att säkerställa att
	interferens som orsakas av en station
	i amatörsatellittjänsten kan stoppas
	omedelbart.\vspace{1ex}\\
	
\end{tabularx}

\vspace{1em} \hrule \vspace{1em}

\emph{Kom ihåg:}

\begin{itemize}
	\item vid nödsituation och för katastrofhjälp får amatörradion användas i radiotrafik för tredje parts räkning.
\end{itemize}

\emph{Lär dig:}

\begin{itemize}
	\item att sändning mellan amatörstationer i olika länder ska ske på vårdat språk
	\item att under sändningspassen ska amatörradiostationerna sända sina anropssignaler med korta intervaller
\end{itemize}


