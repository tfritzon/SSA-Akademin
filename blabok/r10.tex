Kapitlet ”Dagbok” ingår inte i Post- och
telestyrelsens krav för att bli sändaramatör.

Det finns i dag inget krav på att en radioamatör skall föra loggbok på sina förbindelser. Det finns ändå flera goda skäl till att föra en loggbok när du använder din sändare. 


\section{Radiostörningar}

Genom att föra en loggbok så kan du kontrollea om du har sänt vid tidpunkter som t.ex. grannar har haft störningar på sin utrustning. Detta kan vara viktigt för grannsämjan, särskilt som många kan se med viss skepsis på antenner som sätts upp av sändaramatören.

\section{Vågutbredningsfenomen}

Konditionerna på de olika frekvensbanden varierar och går i olika cykler. Genom att se tillbaka i loggboken kan man bilda sig en uppfattning om vilka tider på dygnet, året och med tiden i solcykeln vilka frekvensband lämpar sig bäst från din plats att nå andra platser.

\section{QSL-kort}

Sändaramatörer skickar QSL-kort (kvittenskort) som bekräftelser på en radiokontakt. Din loggbok är förstås grundmaterialet för att kontrollera inkommande och skicka ut sådana kvittenser.

I dag körs ofta detta digitalt med dator, exempelvis via tjänster som eQSL eller LoTW (loggbook of the world, ARRL:s loggsystem).

\section{Diplom och tävlingar}

För att kvalificera sig i olika tävlingar eller för vissa diplom behöver man kunna styra sina kontakter genom loggböcker och kanske QSL-kort. 

ARRL, den amerikanska amatörradioorganisationen, utger bl.a. DXCC-diplomet,
där kravet är att ha QSL-kort från minst 100 olika radioländer (DXCC-områden).
Amatörradioföreningar eller radioklubbar anordnar även tävlingar (contests). Det går
ut på, att under en bestämd tidsperiod, ha kontakt med så många sändaramatörer som
möjligt. Tävlingstiden kan variera från en timme till flera timmar, ett dygn, eller till och med 48 timmar. Tävlingarna pågår normalt under helgerna. Underlaget för diplomen och tävlingarna utgörs av QSL-korten, eller ibland endast utdrag ur din loggbok.

I dag sköts också detta digitalt via LoTW eller liknande.

Vid tävlingar skickar man in en särskild tävlingslogg. För olika tävlingar gäller lite olika regler för format och hur den skall skickas in. För att få poäng i tävlingarna korsrefereras sedan loggarna så att man ser att de stationer du pratat med också har dig i sina loggar.

\section{Loggens beståndsdelar}

Det finns färdiga loggblad man kan köpa från SSA, man kan också göra egna och skriva ut, färdiga PDF-filer kan enkelt finnas på nätet.

\begin{itemize}
	\item Datum och klockslag i UTC
	\item Motstationens anropssignal
	\item Mottagen och skickad signalrapport (RS/RST)
	\item Frekvensband
	\item Sändningsslag
	\item Den egna effekten
	\item Motstationens namn, QTH, ev. lokatorkoder m.m.
	\item Kommentarer
\end{itemize}

\section{Dataloggning}

Det finns många olika loggprogram för
persondatorer. Många är helt kostnadsfria
och kan hämtas på Internet. Samtliga loggprogram har de uppgifter, som bör finnas med
i dagboken.

De flesta loggprogram har oftast en mängd
funktioner, utöver att vara loggbok. Det finns
t.ex. funktioner, för att styra inställningar av
radiostationen och för att sända och ta emot
telegrafi eller digitala trafiksätt.

\section{Rapportkod (RST-koden)}
Sändaramatörer använder RST-koden, för
att ange hur motstationen hörs.

\begin{itemize}
	\item R står för läsbarhet (readability) och
	graderas från 1 till 5 där 1 är oläsligt och 5 fullt läsbart
	\item S står för signalstyrka (signal strength)
	och graderas från 1 till 9 där 1 är lägst och 9 högst
	\item T står för tonkvalitet (tone) och graderas
	från 1 till 9 där 1 är riktigt dåligt och 9 mycket bra
\end{itemize}

Vid en telefoniförbindelse anges endast
läsbarhet och signalstyrka.

\subsection{R-skalan}

Här anger man uppfattbarheten (readability) på en skala från 1 till 5.

\begin{itemize}
	\item [1] Oläsbar
	\item [2] Knappt läsbar, enstaka ord urskiljbara
	\item [3] Läsbar med stor svårighet
	\item [4] Läsbar med obetydlig svårighet
	\item [5] Helt läsbar
\end{itemize}

\subsection{S-skalan}

På en mottagare eller sändtagare (transceiver) har som regel en signalnivåmätare. Genom att läsa av den får man ett signalvärde från 1 till 9.

Man kan också bedöma med örat efter följande skala:

\begin{itemize}
	\item [1] Signalerna nätt och jämt uppfattbara
	\item [2] Mycket svaga signaler
	\item [3] Svaga signaler
	\item [4] Något svaga signaler
	\item [5] Ganska goda signaler
	\item [6] Goda signaler
	\item [7] Mycket goda signaler
	\item [8] Starka signaler
	\item [9] Mycket starka signaler
\end{itemize}

Om man pratar signalstyrka i decibel (dB) så skall man känna till att det är 6 dB mellan varje skalsteg i S-skalan. Det betyder att en S8 är \SI{12}{dB} starkare än en S6.

% Layout
\newpage

\subsection{T-skalan}

Tonskalan används som ett kvalitetmått på telegrafi och bedöms enligt följande:

\begin{itemize}
	\item [1--6] Används mycket sällan
	\item [7] Nästan ren ton, ostabil med tydligt brum
	\item [8] Nästan ren ton med spår av brum eller ojämnhet
	\item [9] Helt ren ton, stabil
\end{itemize}

\vspace{1em} \hrule \vspace{1em}

\emph{Kom ihåg:}

\begin{itemize}
	\item Noteringar i loggboken är en viktig informationsbas för dig
	\item För diplom och liknande är loggbok och QSL-kort viktig information
\end{itemize}

\emph{Lär dig:}

\begin{itemize}
	\item Vilka uppgifter som ingår i en loggbok
\end{itemize}