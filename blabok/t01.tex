\section{Ström}

Elektrisk ström består av elektroner, som
förflyttar sig genom en ledning. Elektronerna
är bärare av elektriska laddningar. Ju fler
elektroner, desto högre ström. Elektronerna
förflyttar sig i motsatt riktning mot strömmen.
Strömstyrka mäts med enheten \emph{Ampere,}
förkortat \enhet{A}.

%%% Bild strömriktning -- elektroners riktning

\[
\begin{array}{rclcl}
\SI{1}{A}  &=& \SI{1000}{mA} \\
\SI{1}{mA} &=& \SI{0.001}{A} \\
\SI{1}{\mu A} &=& \SI{0.001}{mA} &=& \SI{0.000001}{A}
\end{array}
\]

Som symbol för ström används \vbokstav{I}.

\section{Spänning}

För att få elektroner att förflytta sig gen\-om
en ledning behövs en drivande kraft. Denna
kraft kallas den el\-ek\-tro\-mo\-to\-ris\-ka kraften
eller spänning. Spänningen kan jämföras
med trycket i en vattenledning. Ju högre
tryck i ledningen, desto mer vatten kommer
det ur kranen. Ju högre spänning vi har, desto
mer ström flyter det genom en ledning.
Spänning mäts i enheten \emph{Volt,} förkortat \enhet{V}.

\[
\begin{array}{rcl}
\SI{1}{kV} &=& \SI{1000}{V} \\
\SI{1}{mV} &=& \SI{0.001}{V} \\
\SI{1}{\mu V} &=& \SI{0.001}{mV}
\end{array}
\]

Som symbol för spänning används \vbokstav{U}.

\section{Resistans -- motstånd}

Hur mycket ström det flyter i en ledning
vid en given spänning beror på de egenskaper
som ledaren har, d.v.s. hur lätt strömmen
flyter. En tunn ledning gör det svårare för
strömmen att flyta. Ledaren erbjuder större
motstånd än en grov ledare. Resistans mäts i enheten Ohm -- \enhet{\Omega}
(grekiska bokstaven omega).

\[
\begin{array}{rclcl}
\SI{1}{k\Omega} &=& \SI{1000}{\Omega} \\
\SI{1}{M\Omega} &=& \SI{1000}{k\Omega} &=& \SI{1000000}{\Omega}
\end{array}
\]

Som symbol för resistans används \vbokstav{R}.

\section{Grundenheter}

Enheterna \enhet{A}, \enhet{V} och \enhet{\Omega} är grundenheter.

\subsection{Definition}

Definitionen för dessa grundenheter är att
vid spänningen $U = \SI{1}{Volt}$ och resistansen
$R = \SI{1}{Ohm}$, kommer det att flyta en ström
$I = \SI{1}{Ampere}$.

%%% Bild: spänning över och ström genom resistor

\section{Ohms lag}

Ohms lag är en grundformel, som kan
utvecklas och användas till att räkna ut det
mesta inom elektronik. Man använder symbolerna: \vbokstav{I}, \vbokstav{U} och \vbokstav{R}.

\[
\begin{array}{rclcl}
  I &=& \frac{U}{R} &\quad& \text{Ampere} \\
  \\
  U &=& R \cdot I &\quad& \text{Volt} \\
  \\
  R &=& \frac{U}{I} &\quad& \text{Ohm}
\end{array}
\]

För att lättare komma ihåg sambandet, kan man förenkla det hela genom nedanstående
figurer. Täck över den symbol du vill beräkna.

%%\smalltikz{\tikz\node[Formeldreieck]{$\dfrac{U}{R\cdot I}$};}{Formeltriangel för ohms lag.}{fig:ohmslag}

\smalltikz{
\begin{tikzpicture}%%BASE
  \path (0,0) node [regular polygon,regular polygon sides=3,draw,minimum size=2cm] {};
%%  \filldraw[fill=gray] (0,0.133) -- (0,-.499) -- (.86,-.499) -- (.5,0.133) -- cycle;
  \draw (-0.499,0.133) |- (0.499,0.133);
  \path (0,0.7) node [anchor=north] {$U$};
  \path (-0.3,0.05) node [anchor=north] {$R$};
  \path (0.3,0.05) node [anchor=north] {$I$};
  \draw (0,0.133) -- (0,-0.5);
  \path (0,0) node [anchor=north,fill=white] {$\cdot$};
  \vspace{1ex}
%%  \node[align=center,anchor=north] at (0,-1) { };
  \node[align=center,anchor=north,text=white] at (0,-1) {$R=\dfrac{U}{I}$};
\end{tikzpicture}
\hspace{1ex}
\begin{tikzpicture}%%U=RI
  \path (0,0) node [regular polygon,regular polygon sides=3,draw,minimum size=2cm] {};
  \filldraw[fill=gray] (-.5,0.133) -- (0,1) -- (.5,0.133) -- cycle;
  \draw (-0.499,0.133) |- (0.499,0.133);
  \path (0,0.7) node [anchor=north] {$U$};
  \path (-0.3,0.05) node [anchor=north] {$R$};
  \path (0.3,0.05) node [anchor=north] {$I$};
  \draw (0,0.133) -- (0,-0.5);
  \path (0,0) node [anchor=north,fill=white] {$\cdot$};
  \node[align=center,anchor=north,text=white] at (0,-1) {$R=\dfrac{U}{I}$};
  \node[align=center,anchor=north] at (0,-1.25) {$U=R\cdot I$};
\end{tikzpicture}
\hspace{1ex}
\begin{tikzpicture}%%I=U/R
  \path (0,0) node [regular polygon,regular polygon sides=3,draw,minimum size=2cm] {};
  \filldraw[fill=gray] (0,0.133) -- (0,-.499) -- (.86,-.499) -- (.5,0.133) -- cycle;
  \draw (-0.499,0.133) |- (0.499,0.133);
  \path (0,0.7) node [anchor=north] {$U$};
  \path (-0.3,0.05) node [anchor=north] {$R$};
  \path (0.3,0.05) node [anchor=north] {$I$};
  \draw (0,0.133) -- (0,-0.5);
%%  \path (0,0) node [anchor=north,fill=white] {$\cdot$};
  \node[align=center,anchor=north] at (0,-1) {$I=\dfrac{U}{R}$};
\end{tikzpicture}
\hspace{1ex}
\begin{tikzpicture}%%R=U/I
  \path (0,0) node [regular polygon,regular polygon sides=3,draw,minimum size=2cm] {};
  \filldraw[fill=gray] (0,0.133) -- (0,-.499) -- (-.86,-.499) -- (-.5,0.133) -- cycle;
  \draw (-0.499,0.133) |- (0.499,0.133);
  \path (0,0.7) node [anchor=north] {$U$};
  \path (-0.3,0.05) node [anchor=north] {$R$};
  \path (0.3,0.05) node [anchor=north] {$I$};
  \draw (0,0.133) -- (0,-0.5);
%%  \path (0,0) node [anchor=north,fill=white] {$\cdot$};
  \node[align=center,anchor=north] at (0,-1) {$R=\dfrac{U}{I}$};
\end{tikzpicture}
}{Formeltriangel för ohms lag.}{fig:ohmslag}

%% \begin{tikzpicture}
%%   \draw (0,0) -- (1,1.5);
%%   \draw (1,1.5) -- (2,0);
%%   \draw (0,0) -- (2,0);
%%   \draw (0.45,0.66) -- (1.55,0.66);
%%   \draw (1,0) -- (1,0.66);
%%   \node at (1,1) {U};
%%   \node at (0.66,0.25) {R};
%%   \node at (1.25,0.25) {I};
%% \end{tikzpicture}
