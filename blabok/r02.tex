\section{Q-koden}

Q-koden är internationell och har därför fördelen att den kan försås
av alla oberoende av vilket språk som använde.

Förkortningar och Q-koder var ursprungligen tänkta att användas vid
morsetelegrafering för att vinna tid. Numera används Q-koden vit
telegrafi, telefoni och vid digitala trafiksätt.

Q-koden består av bokstaven Q följd av ytterligare två bokstäver till
exempel QSL som betyder ``Jag kvitterar''.

Samma Q-förkortning kan användas som fråga: ``QSL?'' som betyder ``Kan
du ge mig kvittens?''

Andra Q-förkortningar kan användas tillsammans med t.ex. ett
klockslag, en ort, en frekvens m.m.

Ett klockslag:
\begin{itemize}
  \item QRX? -- När anropar du mig igen?
  \item QRX 2048 UTC -- Jag anropar dig igen klockan 20:48 UTC
\end{itemize}

En ort:

\begin{itemize}
\item QTH? -- Vilken är din geografiska position?
\item QTH KARLSBORG -- Min geografiska position är Karlsborg.
\end{itemize}

\scriptsize
\begin{table}[h]
  \begin{tabular}{ll}
    Q-kod & Betydelse som fråga och som svar\\ \hline

    QRK?  & Vilken uppfattbarhet har mina signaler?\\
    QRK   & Uppfattbarh. hos dina (eller *:s signaler) är:\\
          & 1. Dålig\\
          & 2. Bristfällig\\
          & 3. Ganska god\\
          & 4. God\\
          & 5. Utmärkt\\ \hline

    QRM?  & Är min sändning störd av annan station?\\
    QRM   & Störningarna på din signal är:\\
          & 1. Obefintliga\\
          & 2. Svaga\\
          & 3. Måttliga\\
          & 4. Starka\\
          & 5. Mycket starka\\ \hline

    QRN?  & Besväras du av atmosfäriska störningar?\\
    QRN   & Atmosfäriska störningar är:\\
          & 1. Obefintliga\\
          & 2. Svaga\\
          & 3. Måttliga\\
          & 4. Starka\\
          & 5. Mycket starka\\ \hline

    QRO? & Ska jag öka sändareffekten?\\
    QRO  & Öka sändareffekten \\ \hline

    QRP? & Ska jag minska sändareffekten? \\
    QRP  & Minska sändareffekten \\ \hline

    QRS? & Skall jag minska sändningstakten?\\
    QRS  & Minska sändningstakten \\ \hline

    QRQ? & Skall jag öka sändningstakten?\\
    QRQ  & Öka sändningstakten\\ \hline

    QRT? & Skall jag avbryta sändningen?\\
    QRT  & Avbryt sändningen \\ \hline

    QRV? & Är du redo?\\
    QRV  & Jag är redo\\ \hline

    QRX? & När anropar du mig igen?\\
    QRX  & Jag anropar dig igen kl ... (på ... MHz/kHz)\\ \hline

    QRZ? & Vem anropar mig?\\
    QRZ  & Du anropas av ... (på ... kHz/MHz)\\ \hline

    QSB? & Varierar min signalstyrka?\\
    QSB  & Din signalstyrka varierar\\ \hline

    QSL? & Kan du ge mig kvittens?\\
    QSL  & Jag kvitterar\\ \hline

    QSO? & Kan du få förbindelse med ...?\\
    QSO  & Jag kan få förbindelse med ...\\ \hline

    QSY? & Ska jag gå över till annan frekvens?\\
    QSY  & Jag byter frekvens till ... kHz/MHz\\ \hline

    QTH? & Vilken är din geografiska position?\\
    QTH  & Min geografiska position är ...\\ \hline
  \end{tabular}
  \caption{Q-koden}
  \label{tab:q-koden}
\end{table}
\normalsize

\emph{Kom ihåg:}
För att kunna betrakta dig som sändaramatör måste du kunna de allra
vanligaste Q-förkortningarna.

Q-förkortningarna används både på telegrafi och telefoni och i digital
trafik.

\emph{Lär dig:} De ovanstående Q-förkortningarna ska du kunna
utantill. Var mycket uppmärksam på om Q-förkortningen åtföljs av ett
frågetecken eller inte.

