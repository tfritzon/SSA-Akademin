\section{Trafikförkortningar}

Utöver Q-förkortningarna förekommer även internationella
amatörradioförkortningar, s.k. trafikförkortningar.

De flesta trafikförkortningarna bygger på engelska uttryck och används
framför allt vid telegrafi och digitala trafiksätt. Här följer ett
urval av de vanligaste förkortningarna.

\begin{table}[h]
  \begin{tabularx}{\columnwidth}{lXX}
    \textbf{Förk.} & \textbf{Engelsk bet.} & \textbf{Svensk bet.}\\
    BK & Break & Bryt\\
    & Signal used to interrupt a  transmission in progress
    & Sänds för att avbryta motstationens sändning\\
    CQ & General call to all stations & Allmänt anrop\\
    CW & Continuous wave & Telegrafi\\
    DE & From & Från\\
    & Separates the callsign      & Används vid telegrafi \\
    & of the station called from  & och digitala trafiksätt\\
    & that of the calling station & vid identifiering mellan\\
    &                             & motstationens och din \\
    &                             & egen anropssignal\\
    K & Invtation to send & Kom, jag lyssnar\\
    MSG & Message & Meddelande\\
    PSE & Please & Var vänlig och ...\\
    R & Received & Mottaget\\
    RST & Report system & Rapportsystem\\
    & R = Readability & R = Läsbarhet\\
    & S = Signal strength & S = Signalstyrka\\
    & T = Tone report & T = Tonkvalitet\\
    RX & Receiver & Radiomottagare\\
    TX & Transmitter & Radiosändare\\
    UR & Your & Din\\
  \end{tabularx}
  \caption{Trafikförkortningar}
  \label{tab:trafikforkortningar}
\end{table}

\vspace{1em} \hrule \vspace{1em}

\noindent\emph{Kom ihåg:}\\

\begin{itemize}
\item För att kunna betrakta dig som sändaramatör måste du kunna de allra
vanligaste trafikförkortningarna.

\item Trafikförkortningarna används mest på telegrafi och digitala
trafiksätt.\\
\end{itemize}

\noindent\emph{Lär dig:}\\

\begin{itemize}
\item Ovanstående trafikförkortningar som du måste kunna utantill.
\end{itemize}

\section{Amatörradioförkortningar}

Förklaringar av vissa förkortningar som förekommer i boken:

\begin{table*}[h]
  \begin{tabular}{lll}
    Förk. & Engelska & Svenska\\
    UTC & Coordinated universal time & Koordinerad universell tid\\
    CW & Continuous waves & Telegrafi\\
    RTTY & Radio teletype & Fjärrskrift\\
    AMTOR & Amateur Teleprinting over Radio & En speciel form av
    RTTY-protokoll\\
    PSK & Phase Shift Keying & Digital modulationsmetod\\
    && En speciell form av RTTY-protokoll\\
    && Exempelvis PSK31, BPSK31, QPSK31\\
    && PSK63, MT63, PSK125, PSK250\\
    AM & Amplitude modulation & Amplitudmodulering\\
    SSB & Single Sideband & Enkelt sidband\\
    & USB = Upper sideband & USB = Övre sidband\\
    & LSB = Lower sideband & LSB = Lägre sidband\\
    FM & Frequency Modulation & Frekvensmodulering\\
    & NBFM = Narrow Band & NBFM = Smalbands-FM\\
    & Frequency Modulation & \\
    SSTV & Slow Scan Television & Smalbands-TV (bildöverföring)\\
    ATV & Amateur Television & Amatörradio-TV\\
    ITU & International Telecommunications & Internationelle
    Teleunionen\\
    & Union & \\
    IARU & International Amateur Radio & Internationella
    Amatörradiounionen\\
    & Union &\\
    CEPT & European Conference of Postal & Samarbetsorganisation av europeiska\\
    & and Telecommunications & Post- och Telekommunikations-\\
    & administration & myndigheter\\
    PTS & & Kommunikationsmyndigheten\\
    && Post- och Telestyrelsen\\
  \end{tabular}
  \caption{Amatörradioförkortningar}
  \label{tab:amatorradioforkortningar}
\end{table*}

