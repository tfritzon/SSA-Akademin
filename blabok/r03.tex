\section{Amatörradioförkortningar}

Utöver Q-förkortningarna förekommer även internationella
amatörradioförkortningar, s.k. trafikförkortningar.

De flesta trafikförkortningarna bygger på engelska uttryck och används
framför allt vid telegrafi och digitala trafiksätt. Här följer ett
urval av de vanligaste förkortningarna.

\begin{table}[h]
  \begin{tabular}{lll}
    Förk. & Engelsk bet. & Svensk bet.\\
    BK & Break & Bryt\\
    & Signal used to interrupt a & Sänds för att avbryta\\
    & transmission in progress  & motstationens sändning\\
    CQ & General call to all stations & Allmänt anrop\\
    CW & Continuous wave & Telegrafi\\
    DE & From & Från\\
    & Separates the callsign      & Används vid telegrafi \\
    & of the station called from  & och digitala trafiksätt\\
    & that of the calling station & vid identifiering mellan\\
    &                             & motstationens och din egen\\
    &                             & anropssignal\\
    K & Invtation to send & Kom, jag lyssnar\\
    MSG & Message & Meddelande\\
    PSE & Please & Var vänlig och ...\\
    R & Received & Mottaget\\
    RST & Report system & Rapportsystem\\
    & R = Readability & R = Läsbarhet\\
    & S = Signal strength & S = Signalstyrka\\
    & T = Tone report & T = Tonkvalitet\\
    RX & Receiver & Radiomottagare\\
    TX & Transmitter & Radiosändare\\
    UR & Your & Din\\
  \end{tabular}
  \caption{Trafikförkortningar}
  \label{tab:trafikforkortningar}
\end{table}

\emph{Kom ihåg:}

För att kunna betrakta dig som sändaramatör måste du kunna de allra
vanligaste trafikförkortningarna.

Trafikförkortningarna används mest på telegrafi och digitala
trafiksätt.

\emph{Lär dig:}

Ovanstående trafikförkortningar som du måste kunna utantill.
