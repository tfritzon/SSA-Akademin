\section{Lagar, föreskrifter och direktiv}

I detta kapitel får du kännedom om lager, föreskrifter och direktiv som reglerar amatörradiotrafik. Det finns inga särskilda lagar eller föreskrifter för amatörradiotrafik.

\subsection{Aktuella lagar}

--- TODO: Kolla relevans och uppdateringar ---

\begin{itemize}
	\item Lagen om elektronisk kommunikation (LEK 2003:389).
	
	\item Yttrandefrihetsgrundlagen, YGL, (1991:1469).
	
	\item Radioutrustningslagen (SFS 2016:392) samt Radioutrustningsförordningen (SFS 2016:394).
	
\end{itemize}

\subsection{Aktuella direktiv}

\begin{itemize}
	\item EMC-direktivet (2014/30/EU) för elektromagnetisk kompatibilitet.
	
	\item Radioutrustningsdirektivet, RED (2014/53/EU), ersätter det tidigare R\&TTY-direktivet.
\end{itemize}

\subsection{Aktuella föreskrifter}

\begin{itemize}
	\item Post- och telestyrelsens föreskrifter om undantag för tillståndsplikt för vissa radiosändare (PTSFS 2018:3).
	
	\item PTS Föreskrift om radioutrustning (PTSFS 2016:5).
\end{itemize}

\section{Vad som regleras?}

\subsection{Lag om elektronisk kommunikation}

Denna lag utgör underlag för de föreskrifter som Post- och telestyrelsen utfärdar för
amatörradiotrafik. Dessutom reglerar lagen vad som gäller vid avlyssning.

Lagen säger att det är tillåtet att avlyssna
ett radiomeddelande, som inte är avsett för
allmänheten, eller ett QSO, men det är förbjudet att föra innehållet i meddelandet vidare.
I lagen finns också straffbestämmelser som bestämmer straffet, om en sändare använ-
des i strid med ett tillståndsvillkor, eller för vidarebefordran av ett avlyssnat medde-
lande.

Om en radiosändare används i brottslig verksamhet, kan denna förklaras förverkad.

\subsection{Yttrandefrihetsgrundlagen YGL}

YGL är en grundlag, som bl.a. reglerar varje
medborgares rätt att i radio och TV offentligen uttrycka tankar, åsikter och känslor.
Det finns också straffbestämmelser i lagenvarför du måste vara försiktig med att inte
förtala, misskreditera, sprida rykten etc.

\subsection{Lag om radio- och teleterminalutrustning}

All kommersiell radioutrustning inkl.
amatörradioutrustningar som säljs ska uppfylla de krav som finns i denna lag.
Det är tillverkarna som ansvarar för att provning sker, enligt fastställda normer.
Amatörradioutrustningar är undantagna från kraven i denna lag, om det rör sig om
byggsatser med lösa delar, eller modifiering av kommersiell utrustning. Detta undantag
möjliggör för en radioamatör att utföra tekniska experiment, utan att behöva köpa
kommersiell utrustning.

\subsection{EMC-direktivet}
Detta direktiv ställer krav på en elektronisk utrustning, att dels inte generera störande radiostrålning, dels kunna motstå störningar från omgivande utrustningar. En kommersiell utrustning som säljs ska vara provad enligt fastställda normer. Uppfylls kraven, ska tillverkaren märka utrustningen med CE-symbolen.

Även i detta direktiv är amatörradioutrustningar som byggs av lösa delar, eller
modifiering av kommersiella produkter undantagna.

\section{Post- och telestyrelsens föreskrifter
	om undantag från tillståndsplikten
	för vissa radiosändare}

Det finns ingen särskild föreskrift för
amatörradiotrafik. Post- och telestyrelsen
har valt att samla all användning av radio-
sändare, som inte kräver tillståndsplikt, i en
enda föreskrift PTSFS 2018:3.

Nedan finns utdrag ur denna föreskrift,
som reglerar kraven för amatörradiotrafik:

\subsection{2 kap. Definitioner}

\begin{tabularx}{\columnwidth}{lX}
	1 § & I dessa föreskrifter avses med
	\textit{amatörradiocertifikat}: kunskapsbevis utfärdat eller godkänt av Post- och telestyrelsen, som utvisar att godkänt kunskapsprov avlagts,
	amatörradiosändare: radiosändare som är avsedd att användas av personer
	som har amatörradiocertifikat, för sändning på frekvenser som är avsedda för
	amatörradiotrafik, \textit{amatörradiotrafik}: icke yrkesmässig radiotrafik för övning, kommunikation
	och tekniska undersökningar, bedriven i personligt radiotekniskt intresse och utan
	vinstsyfte,\vspace{1ex}\\
\end{tabularx}

\subsection{3 kap. Bestämmelser om undantag från tillståndsplikt}

Nedan en tabell över tillåtna effekter. Metod är hur effekten mäts där pep = peak envelope power vid sändarens utgång, erp = utstrålad effekt jämförd med en dipol (0 dBd) och eirp är ekvivalent isotrop effekt jämförd med en tänkt antenn som strålar lika i alla riktningar (0 dBi).

För att få sända mer än 200\,W krävs sedan 1 nov. 2018 ett särskilt tillstånd som kan sökas från PTS. De frekvensband som då påverkas anges i kolumnen ''Tillst.''

\begin{tabular}{rrlr}
	\textbf{Frekvensband} & \textbf{Effekt} & \textbf{Metod} & \textbf{Tillst.} \\
	    137,7--137,8\,kHz &            1\,W & erp            &  \\
	        472--479\,kHz &            1\,W & eirp           &  \\
	      1810--1850\,kHz &          200\,W & pep            &            1\,kW \\
	      1850--1900\,kHz &           10\,W & pep            &  \\
	      1900--1950\,kHz &          100\,W & pep            &  \\
	      1950--2000\,kHz &           10\,W & pep            &  \\
	      3500--3800\,kHz &          200\,W & pep            &            1\,kW \\
	  5351,5--5366,5\,kHz &           15\,W & eirp           &  \\
	      7000--7200\,kHz &          200\,W & pep            &            1\,kW \\
	10\,100--10\,150\,kHz &          150\,W & pep            &  \\
	14\,000--14\,350\,kHz &          200\,W & pep            &            1\,kW \\
	18\,068--18\,168\,kHz &          200\,W & pep            &            1\,kW \\
	21\,000--21\,450\,kHz &          200\,W & pep            &            1\,kW \\
	24\,890--24\,990\,kHz &          200\,W & pep            &            1\,kW \\
	28\,000--29\,700\,kHz &          200\,W & pep            &            1\,kW \\
	      50,0--52,0\,MHz &          200\,W & pep            &  \\
	    144,0--146,0\,MHz &          200\,W & pep            &            1\,kW \\
	    432,0--438,0\,MHz &          200\,W & pep            &            1\,kW \\
	      1240--1300\,MHz &          200\,W & pep            &            1\,kW \\
	      2400--2450\,MHz &         100\,mW & pep            &  \\
	      5,65--5,85\,GHz &          200\,W & pep            &            1\,kW \\
	      10,0--10,5\,GHz &          200\,W & pep            &            1\,kW \\
	   24,00--24,25\,GMHz &          200\,W & pep            &            1\,kW \\
	      47,0--47,2\,GHz &          200\,W & pep            &            1\,kW \\
	       75,5-81,0\,GHz &          200\,W & pep            &            1\,kW \\
	  122,25--123,00\,GHz &          200\,W & pep            &            1\,kW \\
	        134--141\,GHz &          200\,W & pep            &            1\,kW \\
	        241--250\,GHz &          200\,W & pep            &            1\,kW
\end{tabular}

\begin{tabularx}{\columnwidth}{lX}
	14 § & Den som använder en amatörradiosändare ska ha ett amatörradiocertifikat.
	För att få ett amatörradiocertifikat krävs kunskaper i enlighet med Annex 6 i
	CEPT Rekommendation T/R 61-02, Examinering för amatörradiocertifikat, Vilnius 2004, version 5 februari 2016. Undantag från kravet på amatörradiocertifikat gäller för den som under en tidsbegränsad period utbildar sig för att få ett
	sådant certifikat och för den som under en förevisning tillfälligt använder amatörradiosändare, under förutsättning att användningen av radiosändaren sker under
	uppsikt av en innehavare av amatörradiocertifikat.\vspace{1ex}\\
	
	&Den som innehar amatörradiocertifikat ska ha en egen anropssignal. Denna
	framgår av certifikatet, eller tidigare av amatörradiotillståndet. Mottagare- och
	sändarstationens anropssignaler ska sändas i början och i slutet av varje radioförbindelse. Anropssignalerna ska också upprepas med korta mellanrum under
	pågående radioförbindelse. Under de utbildnings- och förevisningstillfällen som
	anges i stycket ovan ska anropssignal användas som tillhör den innehavare av
	amatörradiocertifikat som har uppsikt över användningen av radiosändaren. Vid
	dessa tillfällen får även anropssignal som tillhör den amatörradioförening eller
	institution som anordnar utbildnings- eller förevisningstillfället användas om
	företrädare för föreningen eller institutionen har uppsikt över användningen av
	radiosändaren.\vspace{1ex}\\
	
	&Automatiska amatörradiosändare, till exempel en radiofyr, repeater eller sändare för positionering ska alltid kunna identifieras genom att en anropssignal
	regelbundet sänds med morsetelegrafi, röstmeddelande eller på annat sätt.
	Anropssignalen ska ange vem som är ansvarig för den automatiska sändaren. Den
	som startar eller använder automatiska amatörradiosändare ska ha eget amatörradiocertifikat och ska använda egen anropssignal. Sådan start och användning får
	även utföras av den som inte har amatörradiocertifikat, om det sker under uppsikt
	av en innehavare av amatörradiocertifikat och dennes anropssignal används. 
	\vspace{1ex}\\
\end{tabularx}

\subsection{Prov för amatörradiocertifikat}

Post- och telestyrelsen, PTS, har delegerat provförrättarverksamheten till SSA. Bestämmelser för provverksamheten finns i SSA:s anvisningar som i sin helhet kan fås från SSA:s webbplats ssa.se.

\newpage

\subsection{SSA Anvisar}

SSA anvisar bland annat om följande moment:

\begin{itemize}
	\item om examineringskrav för amatörradiocertifikat
	\item om auktorisation av proförättare för certifikat och för kunskapsbevis om morsetelegrafering
	\item om handläggningsrutiner vid utfärdande av amatörradiocertifikat samt provförrättarauktorisation
\end{itemize}

Anm. (ej del av PTS föreskrifter):

Post- och telestyrelsen har endast
mandat att föreskriva, att amatörradiosändare inte får störa annan
radioanläggning.

Naturligtvis får du heller inte störa
elektronikutrustning, som uppfyller
störtåligheten i EMC-direktivet, vilket
ligger under Elsäkerhetsverkets föreskrifter.

Det är även Elsäkerhetsverket man vänder sig till om amatörradioanläggningen blir störd av konsumentelektronik eller liknande som avger icke tillåtna störningar. I dag har många amatörer problem med t.ex. robotgräsklippares avgränsningsslingor, LED-lampor samt solcellsanläggningar.

\newpage

\emph{Kom ihåg:}

\begin{itemize}
	\item PTS föreskrifter är styrande dokument för amatörradioverksamheten
	\item På PTS och SSA:s webbplatser finns gällande föreskrifter i varje stund
	\item Lagar som också påverkar amatörradion är t.ex. LEK och YGL
\end{itemize}

\emph{Lär dig:}

\begin{itemize}
	\item Vad lagen som gäller lär om avlyssning
	\item Den maximala effekt du får använda på respektive band med eller utan särskilt tillstånd för högre effekt
\end{itemize}