%R1 Fonetiska alfabetet
%R2 Q-koden
%R3 Amatörradioförkortningar
%R4 Internationella nödsignaler
%R5 Anropssignaler
%R6 Frekvenser
%R7 ITU Radioreglemente
%R8 CEPT:s rekommendationer
%R9 Lagar
%R10 Dagbok

Varje kapitel avslutas med två rutor, ”Kom ihåg!” och
”Lär dig...”. Dessa rutor ska enbart ses som en samman-
fattning av det viktigaste i respektive kapitel. Ett antal
övningsfrågor till kapitlen och facit finns längre
bak i boken. Ytterligare övningsfrågor finns på SSA:s
webbplats www.ssa.se

Det totala innehållet i kapitlen utgör underlag för
provfrågorna för amatörradiocertifikat/licens. Prov
avläggs för en av SSA auktoriserad provförrättare.

\chapter{Lektion R1}

\section{Bokstavering}

För att underlätta mottagning av anrops-
signaler, obekanta ord, svårförståeliga namn
eller förkortningar, använder man sig av
bokstavering. Bokstavering innebär, att varje
enskild bokstav har fått ett personnamn.
Personnamnen har valts ut med hänsyn till
hur lätta de är att ur- och särskilja i till exempel
störd miljö.

\subsection{Svenska bokstaveringsalfabetet}

\begin{table}[h]
  \begin{tabular}{ll|ll}
    A & Adam   & P & Petter  \\
    B & Bertil & Q & Quintus \\
    C & Cesar  & R & Rudolf  \\
    D & David  & S & Sigurd  \\
    E & Erik   & T & Tore    \\
    F & Filip  & U & Urban   \\
    G & Gustav & V & Viktor  \\
    H & Helge  & W & Wilhelm \\
    I & Ivar   & X & Xerxes  \\
    J & Johan  & Y & Yngve   \\
    K & Kalle  & Z & Zäta    \\
    L & Ludvig & Å & Åke     \\
    M & Martin & Ä & Ärlig   \\
    N & Niklas & Ö & Östen   \\
    O & Olof   &   &  \\
    &        &   &  \\
    1 & Ett    & 6 & Sexa    \\
    2 & Tvåa   & 7 & Sju     \\
    3 & Trea   & 8 & Åtta    \\
    4 & Fyra   & 9 & Nia     \\
    5 & Femma  & 0 & Nolla
  \end{tabular}
  \caption{Svenska bokstaveringsalfabetet}
  \label{tab:sv-bokstavering}
\end{table}

Om man alltid använder samma namn,
så kan man i en störd miljö gissa sig till
vilken bokstav som menas.
Vid bokstavering uttalas bokstaven A som
ADAM, bokstaven B som BERTIL och så
vidare.

Det finns även ett internationellt boksta-
veringsalfabet (på engelska). Det används
på motsvarande sätt som det svenska
bokstaveringsalfabetet. Internationellt är det
mycket viktigt att inte hitta på andra engelska
ord eller namn, eftersom motstationen kanske
inte kan engelska.

\section{Internationella bokstaveringsalfabetet}
\emph{Internationella bokstaveringsalfabetet här}

Kom ihåg: En korrekt bokstavering underlättar all radiokommunikation

Lär dig: Bokstaveringsalfabetena som du måste kunna utantill!
